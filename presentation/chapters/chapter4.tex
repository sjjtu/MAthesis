\section{Results}

\begin{frame}
    \frametitle{AE-(dp)MERF}

    \begin{columns}
        \begin{column}{0.48\textwidth}
        \begin{figure}
            \centering
            \includegraphics[scale=0.25]{gen_aemerf.png}
        \end{figure}
        \begin{figure}[h]
            \centering
            \includegraphics[scale=0.25]{gen_aedpmerf_eps05.png}
        \end{figure}
    \end{column}
    \begin{column}{0.48\textwidth}
        \begin{figure}
            \centering
            \includegraphics[scale=0.25]{gen_aedpmerf_eps1.png}
        \end{figure}
        \begin{figure}[h]
            \centering
            \includegraphics[scale=0.25]{gen_aedpmerf_eps001.png}
        \end{figure}
    \end{column}
    
    \end{columns}
    \centering
    \textcolor{gold}{Figure:} AE-(dp)MERF generated samples

\end{frame}

\begin{frame}{AE-(dp)MERF: Utility}
    \begin{figure}
        \centering
        \includegraphics[scale=0.23]{results_aedpmerf.png}
        \caption{Results of AE-(DP)MERF with different privacy budgets (lower epsilon means higher privacy)}
        \label{fig:enter-label}
    \end{figure}
\end{frame}

\begin{frame}
    \frametitle{AE-(dp)WGAN}

    \begin{columns}
        \begin{column}{0.48\textwidth}
        \begin{figure}
            \centering
            \includegraphics[scale=0.25]{gen_aewgan.png}
        \end{figure}
        \begin{figure}[h]
            \centering
            \includegraphics[scale=0.25]{gen_aedpwgan_eps35.png}
        \end{figure}
    \end{column}
    \begin{column}{0.48\textwidth}
        \begin{figure}
            \centering
            \includegraphics[scale=0.25]{gen_aedpwgan_eps25.png}
        \end{figure}
        \begin{figure}[h]
            \centering
            \includegraphics[scale=0.25]{gen_aedpwgan_eps5.png}
        \end{figure}
    \end{column}
    
    \end{columns}
    \centering
    \textcolor{gold}{Figure:} AE-(dp)WGAN generated samples

\end{frame}

\begin{frame}{AE-(dp)WGAN: Utility}
    \begin{figure}
        \centering
        \includegraphics[scale=0.23]{restuls_dprtsgan.png}
        \caption{Results of AE-(dp)WGAN with different privacy budgets (lower epsilon means higher privacy)}
        \label{fig:enter-label}
    \end{figure}
\end{frame}

\begin{frame}{Conclusion}
    \begin{itemize}
        \item<1-> \alert{AE-(dp)MERF performs best} and is very efficient computationally.
        \item<2-> AE-(dp)MERF can work in \alert{lower epsilon ranges}, which translates to \alert{stronger privacy guarantees}.
        \item<3-> \alert{AE-(dp)WGAN gives worse utility} and can only work with meaningless privacy budgets $\epsilon$.
        \item<4-> We lose utility when \alert{replacing original data with non-private synthetic} data.
        \item<5-> BUT: Adding \alert{privacy does not further degrade the utility} for anomaly detection too much until too much noise is added.
    \end{itemize}
\end{frame}

\begin{frame}{Contamination}
    We \alert{contaminate} the train set that only consists of regular samples with \alert{1\%, 2\%, 5\% anomalous samples} (the percentage of heartbeat arrhythmias is estimated to be around max. 5\%).
\end{frame}

\begin{frame}{Contamination}

    \begin{figure}[h]
        \centering
        \includegraphics[scale=0.28]{str_comtam.png}
        \caption{Structure of Contamination Experiment}
        \label{fig:enter-label}
    \end{figure}

\end{frame}

\begin{frame}{Contamination: AE-(DP)MERF}
    \begin{columns}
        \begin{column}{0.65\textwidth}
            \begin{figure}

                \centering
                \only<1>{\includegraphics[width=0.7\textheight]{results_aedpmerf_contam.png}}%
                \only<2>{\includegraphics[width=0.7\textheight]{results_aedpmerf_contam_baseline.png}}%
                \only<3>{\includegraphics[width=0.7\textheight]{results_aedpmerf_contam_aemerf.png}}%
                \only<4>{\includegraphics[width=0.7\textheight]{results_aedpmerf_contam_aedpmerf.png}}%
                \only<5>{\includegraphics[width=0.7\textheight]{results_aedpmerf_contam.png}}
        
                \caption{Contaminated training set: AE-(DP)MERF}
        
            \end{figure}
        \end{column}
        \begin{column}{0.34\textwidth}
            \begin{itemize}
                \scriptsize
                \item<2-> \alert{Baseline model} performance degrades with increasing contamination percentage.
                \item<3-> \alert{AE-MERF} generated samples retain stable utility.
                \item<4-> Utility of \alert{AE-dpMERF} generated samples first increases and then decrease when contamination is too high.
                \item<5-> Utility of synthetic data is higher than baseline model    
            \end{itemize}
        \end{column}
    \end{columns}
    
\end{frame}

\begin{frame}
    \frametitle{Contamination: AE-(DP)MERF}

    \textcolor{gold}{Hypothesis: } Noise added during data generation and DP noise can have a \alert{regularising effect} on the synthetic data which counteracts the contamination.

\end{frame}

