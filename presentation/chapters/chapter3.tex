%------------------------------------------------
\section{Privacy-preserving Time Series Data Generation}
%------------------------------------------------

\begin{frame}
    \frametitle{Review: Differential Privacy}
    \onslide<1,2>\textbf{Idea.} \alert{Hide the influence} of one particular sample on the output of the model by \alert{adding randomness}.
    \onslide<2->\begin{definition}[Differential Privacy]
        A randomised algorithm $\mathcal{M}$ is $(\epsilon, \delta)$- differentially private if for all  set of outcomes $S \subset ran \mathcal{M}$ and for all databases $x,y $, such that they \textbf{only differ in one element}, we have
        \begin{align}
            \mathbb{P}(\mathcal{M}(x) \in S) \le e^\epsilon \cdot \mathbb{P}(\mathcal{M}(y) \in S) + \delta \mcomma
        \end{align}
        where the probability is taken over the randomness of $\mathcal{M}$.
    \end{definition}

    \onslide<3>\textbf{Informally.} Replacing one record in the data will not change the outcome of algorithm $\mathcal{M}$ \textit{too much} (specified via privacy budget $\epsilon$). The lower $\epsilon$ the stricter the privacy guarantees.

\end{frame}

\begin{frame}
    \frametitle{Examples of DP mechanism}

    \begin{itemize}
        \item Gaussian mechanism
        \begin{itemize}
            \item Add \alert{Gaussian noise} to output of some function.
            \item For a given function  $f:\mathbb{N}^{|\mathcal{X}|} \longrightarrow \mathbb{R}^d$, privacy parameters $\epsilon \in (0,1)$ and $\delta>0$ define the gaussian mechanism $F(x)$ as follows:
            \begin{align}
                F(x) = f(x) + \mathcal{N}(0, \sigma^2)
            \end{align}
            where the variance is calibrated to satisfy DP.
        \end{itemize}
        \item DP-SGD
        \begin{itemize}
            \item DP version for training neural networks
            \item \alert{Add noise to gradients} while training
        \end{itemize}
    \end{itemize}
    

\end{frame}

\begin{frame}{Models}
    \onslide<1->{\textcolor{tomato}{AE-(dp)MERF}}
    \begin{itemize}
        \item<1-> AE-(dp)MERF is based on DP-MERF (best state of the art generator for tabular data).
        \item<2-> Simple architecture with mathematically sophisticated loss function.
        \item<3-> Does not work with time series data out of the box, but we will modify it so it works.
    \end{itemize}
    \onslide<4->{\textcolor{tomato}{AE-(dp)WGAN}}
    \begin{itemize}
        \item<5-> Model based on GAN network, which are commonly used in image generation.
        \item<6-> Based on RTSGAN, which delivers state of the art performance for time series data.
        \item<7> No private counterpart, hence we will implement our own private version.
    \end{itemize}
    
\end{frame}

