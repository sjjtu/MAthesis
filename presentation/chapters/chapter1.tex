\section{Project introduction}
\begin{frame}{Anomaly detection using privacy-preserving, synthetic time series data}

    \begin{itemize}
        \item ECG data can used for different downstream tasks like arrhythmia detection or other cardiac diseases and even studies on sleep, emotions and stress.
        \item Heartbeat data can be used for biometrics authentication similar to the fingerprint. Hence, ensuring privacy is important.
        \item Synthetic data with differential private (DP) guarantees is a promising solution to ensure privacy independent of downstream task.
        \item Commonly, a gain in privacy results in a loss of utility.
        \item For anomaly detection this might not be the case (?).
    \end{itemize}
    Goal: generate useful and privacy-preserving ECG data for anomaly detection.
\end{frame}

\begin{frame}{Structure}
    \begin{enumerate}
        \item Train baseline model for anomaly detection only on regular heartbeat data using an LSTM-AE.
        \item Generate heartbeat data (without DP) using two approaches:
        \begin{itemize}
            \item[--] AE-MERF
            \item[--] RTSGAN
        \end{itemize}
        \item Train LSTM-AE for anomaly detection on synthetic data and test on real (TSTR).
        \item Add DP noise and repeat:
        \begin{itemize}
            \item[--] AE-DPMERF
            \item[--] DP-RTSGAN
        \end{itemize}
        \item Contaminate training data with anomalous heartbeats and repeat
    \end{enumerate}
\end{frame}


