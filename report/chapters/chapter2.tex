In diesem Kapitel betrachten wir lineare, hyperbolische, partielle Differentialgleichungen folgender Gestalt:


\begin{equation}
    \begin{cases}
        u_t = Au_x+Bu_y \qquad A,B \in Mat_{n,n}(\mathbb{C}) \text{ hermitesch} \\
        u(x,y,0) = f(x,y) \in\operatorname{L^2}(-\infty, \infty)
    \end{cases}
\end{equation}
Wir suchen eine vektorwertige Funktion 
\begin{align}
    u(x,y,t)=(u_1(x,y,t), u_2(x,y,t), ... , u_n(x,y,t)) \; ,
\end{align}
die das Anfangswertproblem löst.

Wir nehmen an, dass für jedes $f \in \operatorname{L^2}(-\infty, \infty)$ eine eindeutige Lösung existiert. \parencite[][Kapitel 13.3]{richtmyer1994difference}
Zuerst folgt eine kurze Einführung in die benötigte Theorie aus der Numerik. Anschließend wird ein Differenzenschema, das sogenannte Lax-Wendroff-Verfahren, hergeleitet und die Konvergenz des Verfahrens untersucht.

\subsection{Differenzenverfahren}

Anstatt alle Punkte auf dem betrachteten Gebiet $G$ zu untersuchen, werden nur endlich viele Punkte untersucht. Eine gängige Prozedur zur Auswahl dieser Punkte ist, ein Gitter über das Gebiet zu legen und die Gitterpunkte zu betrachten. Der Einfachheit halber betrachten wir nur äquidistante Gitter mit einer konstanten Schrittweite $h$ auf rechteckigen Gebieten der Form $[a,b] \times [c,d] \in \mathbb{R}^2$. Die nun folgenden grundlegenden Definitionen sind \parencite[][Kapitel 5]{bollhofer2013numerische} entnommen.

\begin{definition}(Gitter)
    Seien Schrittanzahlen $n_x, n_y \in \mathbb{N}$ gegeben und Schrittweiten $h_x = \frac{b-a}{n_x}$ und $h_y = \frac{d-c}{n_y}$. Dann ist ein Gitter von $[a,b] \times [c,d] \in \mathbb{R}^2$ mit Schrittweiten $h_x$ und $h_y$ gegeben durch die Menge:
    \begin{align}
        \left\{ (x_i, y_j) \in \mathbb{R}^2 \; | \; x_i=a+ih_x, \; y_j=b+jh_y \; ,  i=0, ..., n_x , j=0,..., n_y \right\}
    \end{align}

\end{definition}

\begin{ex}
Wir diskretisieren die Menge $[0,1] \times [0,1]$ mit Schrittanzahlen in x-Richtung $n_x = 6$ und in y-Richtung $n_y=3$. Dies liefert ein Gitter mit 28 Gitterpunkten.
\begin{figure}[H]\phantom{asd}
    \caption{Diskretisierung von $[0,1]\times[0,1]$}
    \centering
    \begin{tikzpicture}
        \filldraw [black] (0,0) circle (2pt);
        \filldraw [black] (1,0) circle (2pt);
        \filldraw [black] (2,0) circle (2pt);
        \filldraw [black] (3,0) circle (2pt);
        \filldraw [black] (4,0) circle (2pt);
        \filldraw [black] (5,0) circle (2pt);
        \filldraw [black] (6,0) circle (2pt);

        \filldraw [black] (0,2) circle (2pt);
        \filldraw [black] (1,2) circle (2pt);
        \filldraw [black] (2,2) circle (2pt);
        \filldraw [black] (3,2) circle (2pt);
        \filldraw [black] (4,2) circle (2pt);
        \filldraw [black] (5,2) circle (2pt);
        \filldraw [black] (6,2) circle (2pt);

        \filldraw [black] (0,4) circle (2pt);
        \filldraw [black] (1,4) circle (2pt);
        \filldraw [black] (2,4) circle (2pt);
        \filldraw [black] (3,4) circle (2pt);
        \filldraw [black] (4,4) circle (2pt);
        \filldraw [black] (5,4) circle (2pt);
        \filldraw [black] (6,4) circle (2pt);

        \filldraw [black] (0,6) circle (2pt);
        \filldraw [black] (1,6) circle (2pt);
        \filldraw [black] (2,6) circle (2pt);
        \filldraw [black] (3,6) circle (2pt);
        \filldraw [black] (4,6) circle (2pt);
        \filldraw [black] (5,6) circle (2pt);
        \filldraw [black] (6,6) circle (2pt);
        
        \draw[color=gray] (0,0) -- (6,0);
        \draw[color=gray] (0,0) -- (0,6);
        \draw[color=gray] (6,0) -- (6,6);
        \draw[color=gray] (0,6) -- (6,6);

        \node at (-1,-1) {0};
        \node at (7,-1) {1};
        \node at (-1,7) {1};

        \draw (0, -1) -- (6, -1);
        \draw (-1, 0) -- (-1, 6);
        \node at (-1.5, 3) {y};
        \node at (3, -1.5) {x};

        \node at (0, -.5) {\small $(x_0, y_0)$};
        \node at (3, -.5) {\dots};
        \node at (6, -.5) {\small $(x_6, y_0)$};

        \node at (0, 1.5) {\small $(x_0, y_1)$};
        \node at (3, 1.5) {\dots};
        \node at (6, 1.5) {\small $(x_6, y_1)$};

        \node at (0, 3.5) {\small $(x_0, y_2)$};
        \node at (3, 3.5) {\dots};
        \node at (6, 3.5) {\small $(x_6, y_2)$};

        \node at (0, 5.5) {\small $(x_0, y_3)$};
        \node at (3, 5.5) {\dots};
        \node at (6, 5.5) {\small $(x_6, y_3)$};

    \end{tikzpicture}
\end{figure}
\end{ex}

Analog kann man auch die Zeitachse diskretisieren. Da aber Grenzwerte der Zeit $t$ bis in die Unendlichkeit betrachtet werden, wird ein eindimensionales, unendliches Gitter über die Zeitachse gelegt.

\begin{definition}(Diskretisierung der Zeit)
    Sei $t > 0$. Dann ist eine Diskretisierung der Zeit mit der Schrittweite $h_t$ gegeben durch 
    \begin{align}
        \left\{t_k \; | \; t_k = k h \; , \; k \in \mathbb{N} \right\}
    \end{align}    
\end{definition}

Nachdem die Zeit- und Raumvariablen durch ihre Diskretisierung auf einem Gitter ersetzt wurden, gilt es nun die Ableitungen mithilfe sogenannter Differenzengleichungen in eine diskrete Form zu überführen. Betrachtet man dafür eine hinreichend glatte Funktion $u(x,y)$ an einem beliebigen Punkt $(x_0, y_0) \in \mathbb{R}^2$, dann folgt mithilfe einer Taylorentwicklung für einen benachbarten Punkt $(x_0 \pm h_x, y_0)$ im Gitter in x-Richtung mit Schrittweite $h_x$:

\begin{align*}
    u(x_0 \pm h_x, y_0) &= u(x_0, y_0) + \partial_x u(x_0, y_0)(x_0 \pm h_x -x_0) \\
    &\phantom{= u(x_0, y_0)i} + \frac{1}{2} \partial^2_x u(x_0, y_0)(x_0 \pm h_x -x_0)^2 + \dots  \\
    &=  u(x_0, y_0) \pm \partial_x u(x_0, y_0)h_x + \mathcal{O}(h_x^2)
\end{align*}

\begin{align}
    \Leftrightarrow \frac{u(x_0 \pm h_x, y_0) - u(x_0, y_0)}{h_x} + \mathcal{O}(h_x) =  \pm \partial_x u(x_0, y_0)
\end{align}

Dann ergeben sich folgende Möglichkeiten für eine Näherung der ersten Ableitung von $u$ in $x$:
\begin{center}
    \renewcommand{\arraystretch}{3}
    \begin{tabular}{c l}
        \label{eq:diff_forward}$\displaystyle \partial_x u(x_0, y_0) = \frac{u(x_0+h_x, y_0) - u(x_0, y_0)}{h_x} + \mathcal{O}(h_x)$ & Vorwärtsdifferenzenquot. \\
        \label{eq:diff_backward}$\displaystyle \partial_x u(x_0, y_0) = \frac{u(x_0, y_0) - u(x_0 - h_x, y_0)}{h_x} + \mathcal{O}(h_x)$& Rückwärtsdifferenzenquot. \\
        \label{eq:diff_central}$\displaystyle \partial_x u(x_0, y_0) = \frac{u(x_0 + h_x, y_0) - u(x_0 - h_x, y_0)}{2h_x} + \mathcal{O}(h_x^2)$ & zentraler Differenzenquot. 
    \end{tabular}
\end{center}

Wobei der zentrale Differenzenquotient aus der Summe von Vorwärts- und Rück\-wärts\-differenzenquotienten gebildet wird. Analog können nun die Näherungen für die erste Ableitung in $y$-Richtung bestimmt werden. Auch die zweite Ableitung lässt sich mithilfe der Taylorentwicklung annähern. Dafür wird erst wieder eine Taylorentwicklung bis zur dritten Ableitung durchgeführt.

\begin{align} \label{eq:diff_quot_add}
    u(x_0 + h_x, y_0) &= u(x_0, y_0) + \partial_x u(x_0, y_0)(x_0 + h_x - x_0) \nonumber \\ 
    & + \frac{1}{2} \partial^2_x u(x_0, y_0)(x_0+ h_x -x_0)^2 + \frac{1}{6} \partial^3_x u(x_0, y_0) (x_0 + h_x -x_0)^3 + \mathcal{O}(h_x^4)
\end{align}

\begin{align} \label{eq:diff_quot_diff}
    u(x_0 - h_x, y_0) &= u(x_0, y_0) + \partial_x u(x_0, y_0)(x_0 - h_x - x_0)\nonumber \\ 
    & + \frac{1}{2} \partial^2_x u(x_0, y_0)(x_0 - h_x -x_0)^2 + \frac{1}{6} \partial^3_x u(x_0, y_0) (x_0 - h_x -x_0)^3 + \mathcal{O}(h_x^4) 
\end{align}

Addiert man nun die Gleichungen \ref{eq:diff_quot_add} und \ref{eq:diff_quot_diff}, dann heben sich die Terme der ersten und dritten Ableitung auf und man erhält:

\begin{align}
    u(x_0 + h_x, y_0) + u(x_0 - h_x, y_0) = 2u(x_0, y_0) + h_x^2 \partial_x^2 u(x_0, y_0) + \mathcal{O}(h_x^4)
\end{align}

Somit erhält man eine Näherung der zweiten Ableitung:

\begin{center}
    \begin{tabular}{c l}
        $\displaystyle \partial_x^2 u(x_0, y_0) = \frac{u(x_0 + h_x, y_0) - 2 u(x_0,y_0) + u(x_0 - h_x, y_0) }{h_x^2} + \mathcal{O}(h_x^2)$ & 
        \begin{tabular}{@{}c}
            \small zentr. \\ Differenzenquot. \\ der 2. \\ Ableitung
        \end{tabular}
    \end{tabular}
\end{center}

Analog ergeben sich die partiellen Ableitungen nach $y$ mit Schrittweite $h_y$. \\

Im späteren Verlauf der Arbeit werden wir noch die gemischte zweite Ableitung benötigen, daher berechnen wir diese jetzt auch analog. Dafür wenden wir den zentralen Differenzenquotienten der ersten Ableitung nach $y$ auf den zentralen Differenzenquotienten der ersten Ableitung nach $x$ an:
\begin{align} 
    \partial_y \partial_x u(x_0, y_0) & = \partial_y (\partial_x u)(x_0, y_0) \nonumber \\
    & = \partial_y \left( \frac{u(x_0 + h_x, y_0) - u(x_0 - h_x, y_0)}{2h_x} \right) + \mathcal{O}(h_x^2) \nonumber \\
    & = \frac{\partial_y u(x_0 + h_x, y_0) - \partial_y u(x_0 - h_x, y_0)}{2h_x} \nonumber + \mathcal{O}(h_x^2) \\
    & = \frac{1}{4h_xh_y} \bigl[ u(x_0 + h_x, y_0 + h_y) - u(x_0 + h_x, y_0 - h_y) \nonumber \\ 
        & \phantom{mmmmmmm} - u(x_0 - h_x, y_0 + h_y) + u(x_0 - h_x, y_0 - h_y) \bigr] \nonumber \\
        & \phantom{mm} + \mathcal{O}(h_x^2) + \mathcal{O}(h_y^2) \label{eq:diff_second_mixed}
\end{align}

Nach dem Satz von Schwarz spielt es für hinreichend glatte Funktionen keine Rolle, ob zuerst die Ableitung nach $x$ und dann die Ableitung nach $y$ berechnet wird oder umgekehrt.

\begin{ex}(Finite-Differenzen-Verfahren mit zentralen Differenzenquotienten)
    Im Folgenden betrachten wir wieder das Anfangswertproblem vom Beginn des Kapitels:
    \begin{equation} \label{eq:hyp_lin_pde}
        \begin{cases}
            u_t = Au_x+Bu_y \qquad A,B \in Mat_{n,n}(\mathbb{C}) \text{ hermitesch} \\
            u(x,y,0) = f(x,y) \in\operatorname{L^2}(-\infty, \infty)
        \end{cases}
    \end{equation}

    Wir leiten nun das Lax-Wendroff-Schema nach \parencite[][S. 277]{goldberg1982numerical} her.

    Falls $u$ eine Lösung ist, dann ergibt eine Taylorentwicklung in den Zeitvariablen $t$ um den Entwicklungspunkt $t_0$ für den Zeitpunkt $t_0+h$ wobei $h>0$ zunächst ein beliebiges Zeitinkrement ist:

    \begin{align} \label{eq:lw_pt1}
        u(x,y,t_0+h) & = u(x,y,t_0) + u_t(x,y,t_0) (t_0+h-t_0) \nonumber \\ 
        & \phantom{m} + \frac{1}{2} u_{tt}(x,y,t_0) (t_0+h-t_0)^2 + \mathcal{O}(h^3)  \nonumber \\
        & = u(x,y,t_0) + hu_t(x,y,t_0)+ \frac{1}{2} h^2 u_{tt}(x,y,t_0) + \mathcal{O}(h^3)
    \end{align}

    Mithilfe der Gleichung (\ref{eq:hyp_lin_pde}) können nun auch die Ableitungen von $u$ bezüglich der Zeit ersetzt werden:

    \begin{align}
        u_t & = Au_x+Bu_y \\ \nonumber \\
        u_{tt} & = \partial_t u_t = \partial_t (Au_x+bu_y) = Au_{xt}+bu_{yt} = Au_{tx}+bu_{ty} \nonumber \\
        & = A(Au_{xx} + Bu_{xy}) + B(Au_{xy} + Bu_{yy}) \nonumber \\ 
        & = A^2u_{xx} + ABu_{xy} + BAu_{xy} + B^2u_{yy}
    \end{align}


    Eingesetzt in Gleichung (\ref{eq:lw_pt1}) ergibt sich folgende Gleichung:

    \begin{align}
        u(x,y,t+h) & = u(x,y,t_0) + hAu_x(x,y,t_0)+hBu_y(x,y,t_0) \nonumber \\ 
        & + \frac{1}{2} h^2 A^2u_{xx}(x,y,t_0) + \frac{1}{2} h^2 (AB+BA)u_{xy}(x,y,t_0) \nonumber \\ 
        & + \frac{1}{2} h^2 B^2u_{yy}(x,y,t_0) + \mathcal{O}(h^3)
    \end{align}

    Wir wählen nun Schrittweiten $h_t, h_x, h_y > 0$. Dann können wir die ersten bzw. zweiten Ableitungen in $x$ und $y$ durch den zentralen Differenzenquotienten der ersten bzw. zweiten Ableitung ersetzen:
    \begin{align}
        u(x,y,t+h_t)  =&  u(x,y,t_0) + \frac{h_t}{2h_x} A \bigl[ u(x_0 + h_x, y_0) - u(x_0 - h_x, y_0) \bigr] \nonumber \\ 
        & + \frac{h_t}{2h_y} B \bigl[ u(x_0, y_0 + h_y) - u(x_0, y_0 - h_y) \bigr] \nonumber \\
        & + \frac{1}{2} \frac{h_t^2}{h_x^2} A^2 \bigl[ u(x_0 + h_x, y_0) - 2 u(x_0,y_0) 
        + u(x_0 - h_x, y_0) \bigr] \nonumber \\
        & + \frac{1}{2} \frac{h_t^2}{4h_x h_y} (AB+BA) \bigl[ u(x_0 + h_x, y_0 + h_y) - u(x_0 + h_x, y_0 - h_y) \nonumber \\[-10pt]
        & \phantom{mmmmmmmmmmm} - u(x_0 - h_x, y_0 + h_y) + u(x_0 - h_x, y_0 - h_y) \bigr] \nonumber \\
        & + \frac{1}{2} \frac{h_t^2}{h_y^2} B^2 \bigl[ u(x_0, y_0+ h_y) - 2 u(x_0,y_0) 
        + u(x_0 , y_0- h_y) \bigr] \nonumber \\
        & + \mathcal{O}(h_t^3) + \mathcal{O}(h_x^2) + \mathcal{O}(h_y^2)
    \end{align}
Wir erhalten damit ein Verfahren, mit welchem wir sukzessive für jeden diskreten Zeitschritt die Funktion $u$ auf dem Gitter approximieren können. Dafür definieren wir die Approximation $v \approx u$ und $\lambda:=\frac{h_t}{h_x}$ bzw. $\mu:=\frac{h_t}{h_y}$ (Im Artikel \parencite[][S. 278]{goldberg1982numerical} ist ein Tippfehler; der Bruch wurde dort fälschlicherweise umgekehrt geschrieben \parencite[vgl.][S. 99]{gustafson1997numerical}). Außerdem nehmen wir an, dass die Schrittweiten in den Raumvariablen $h_x, h_y$ Funktionen der Schrittweite in der Zeit $h_t$ sind, die gegen 0 konvergieren für $h_t \rightarrow 0$. Dadurch ist es für die späteren numerischen Untersuchungen ausreichend, nur noch $h_t$ zu betrachten. Dann können wir $v(jh_x, kh_y, nh_t) =: [v^n_{jk}]$ iterativ über folgendes Schema bestimmen:
\begin{align}
    (LW) %\label{finite_difference}
    \begin{cases} 
        [v^0_{jk}] = f(jh_x, kh_y) \\[10pt]
        [v^{n+1}_{j,k}] = [v^n_{jk}] + \frac{1}{2} \lambda A ([v^n_{j+1,k}] - [v^n_{j-1,k}])+ \frac{1}{2} \mu B ([v^n_{j,k+1}] - [v^n_{j,k-1}]) \nonumber \\[7pt]
        \phantom{v^{n+1}_{j,k}] = [v^n_{jk}].} + \frac{1}{2} \lambda^2 A^2 ([v^n_{j+1,k}] - 2 [v^n_{jk}]+ [v^n_{j-1,k}]) \nonumber \\[7pt]
        \phantom{v^{n+1}_{j,k}] = [v^n_{jk}].} + \frac{1}{8} \lambda \mu  (AB+BA) ([v^n_{j+1, k+1}] - [v^n_{j+1,k-1}] - [v^n_{j-1, k+1}] + [v^n_{j-1,k-1}]) \nonumber \\[7pt]
        \phantom{v^{n+1}_{j,k}] = [v^n_{jk}].} + \frac{1}{2} \mu^2 B^2 ([v^n_{j,k+1}] - 2 [v^n_{jk}]+ [v^n_{j,k-1}])
    \end{cases}
\end{align}

\end{ex}

\subsection{Konsistenz, Stabilität und Konvergenz}

Eine wichtige Frage, der wir uns nun widmen wollen, ist, ob das Verfahren (LW) die analytische Lösung hinreichend gut approximiert. Dafür definieren wir zuerst, welchen Kriterien eine hinreichend gute Approximation genügen soll. Wir orientieren uns dabei an der von P. Lax aufgestellten Theorie zur Approximation von Anfangswertproblemen mithilfe von Finiten-Differenzen \parencite[][Kapitel 3]{richtmyer1994difference}.

Sei $\mathcal{B}$ ein abstrakter Banachraum mit Norm $\|\cdot\|$, für den wir einen Ableitungsbegriff definieren können. Wir betrachten nun ein allgemeines Anfangswertproblem:
\begin{align}
    \begin{cases}
        u_t(t) = T u(t) \qquad T \text{ lin. Operator in } \mathcal{B}, \, t>0\\
        u(0) = u_0 \in \mathcal{B}
    \end{cases}\label{eq:awp}
\end{align}
Das heißt, wir suchen eine Familie von Elementen $\{u(t)\}_t \subseteq \mathcal{B}$, die  (\ref{eq:awp}) erfüllt. Statt alle $t>0$, approximieren wir $u$ auf einem Gitter. Dafür führen wir die Schrittweite $h_t>0$ ein und definieren $u^n:=u(nh_t)$. Den linearen Operator approximieren wir durch einen linearen und beschränkten Operator $D(h_t) \in \mathcal{L}(\mathcal{B})$, der von den Schrittweiten $h_t$, nicht aber von $t$ abhängt. Weiterhin wird $D$ so gewählt, dass $Du^n$ nur aus linearen Summen und Differenzen von Funktionswerten besteht \parencite[vgl.][S. 43]{richtmyer1994difference}. Dann nennen wir $D$ einen Finite-Differenzen-Operator und erhalten damit ein allgemeines Finite-Differenzen-Verfahren:
\begin{align*} 
    (FD)
    \begin{cases}
        w^0 = u_0 \\
        w^{n+1} = D(h_t) w^n \; .
    \end{cases}
\end{align*}
 
Im Folgenden wollen wir nun die Begriffe der Konsistenz, Stabilität und Konvergenz für das Verfahren (FD) motivieren und definieren.  \\\\

Angenommen $u$ ist eine hinreichend glatte Lösung von (\ref{eq:awp}). 
Setzen wir diese Lösung $u$ in das Verfahren (FD) ein, so sollte es die Eigenschaft haben, dass für hinreichend kleine Zeitschritte, der Finite-Differenzen-Operator $D$ den ursprünglichen Operator $T$ approximiert, also für ein $t >0$

\begin{align}
    u_t(t) \approx \frac{u(t+h_t) - u(t)}{h_t} \approx \frac{D(h_t) u^n - u^n}{h_t} &\longrightarrow Tu(t) \nonumber \\
    \text{für } h_t &\longrightarrow 0 \; .
\end{align}

Dies führt auf folgende Definition:

\begin{definition}(Konsistenz)
    Das Verfahren (FD) heißt konsistent, falls für alle $t>0$ gilt:
    \begin{align}
        \left\| \frac{D(h_t)u(t) - u(t)}{h_t} - (Tu(t)) \right\| \longrightarrow 0 \qquad \text{für $h_t \longrightarrow$ 0.}
    \end{align}
\end{definition}

Die Konsistenz ist also eine Eigenschaft des Verfahrens, doch sie gibt keine Aussage über die Güte der Approximation. Eine n-fache Anwendung des Verfahrens auf den Anfangswert $u_0 = w^0$ ergibt: \begin{align}
    D(h_t)^n w^0 = w^{n}
\end{align}

Da man die exakte Lösung nicht kennt, sondern mit dem Verfahren approximieren möchte, sollte das Verfahren für beliebig kleine Schrittweiten die Lösung auch beliebig gut approximieren. Wir fordern folgende Konvergenzeigenschaft:

\begin{definition}(Konvergenz)
    Das Verfahren (FD) heißt konvergent, falls gilt:
    \begin{align}
        \left\| D(h_t)^n u_0 - u(n h_t) \right\| \longrightarrow 0 \qquad \text{für $h_t \longrightarrow$ 0.}
    \end{align}
\end{definition}

Außerdem wollen wir ausschließen, dass sich kleine Rechenfehler im Verfahren nicht beliebig groß potenzieren und es keinen "Blow-Up" gibt. Daher muss für alle $n\in \mathbb{N}$ gewährleistet sein:
\begin{align}
    \left\| D(h_t)^n v^0  \right\| \le \left\| D(h_t)^n \right\| \left\| v^0  \right\| < \infty \nonumber \\ \nonumber \\
    \Leftrightarrow \| D(h_t)^n \| < \infty 
\end{align}

Das führt auf die Definition der Stabilität:

\begin{definition}
    Das Verfahren (FD) heißt stabil, falls für ein $\tau > 0 $ gilt:
    \begin{align}
        \| D(h_t)^n \| < \infty \qquad \text{für $0 < h_t < \tau$.}
    \end{align}
\end{definition}

Es gilt folgende Beziehung zwischen den drei Begriffen:

\begin{thm}[Äquivalenzsatz von Lax] \label{thm:equival_lax}
    Das Verfahren (FD) ist genau dann konvergent, wenn es konsistent und stabil ist.
\end{thm}
\begin{proof}
    siehe \parencite[][Kapitel 3.5]{richtmyer1994difference}
\end{proof}

\subsection{Konvergenz des Lax-Wendroff-Verfahrens}
Nach dem Äquivalenzsatz von Lax (Theorem \ref{thm:equival_lax}) reicht es also aus, Konsistenz und Stabilität zu zeigen. Dabei folgt die Konsistenz leicht durch eine Taylorentwicklung (die wir ja bereits bei der Herleitung des Verfahrens benutzt haben). 

Um Stabilität zu zeigen führen wir eine Stabilitätsanalyse nach von-Neumann durch \parencite[][Kapitel 4.4]{richtmyer1994difference}. Dafür betrachten wir den fouriertransformierten Finite-Diffe\-renzen-Operator und zeigen Stabilität für das transformierte System:
\begin{align}
    G = G(\xi, \eta, \lambda, \mu) = I
    & + \frac{1}{2} \lambda A (e^{i\xi} - e^{-i\xi})+ \frac{1}{2} \mu B (e^{i\eta} - e^{-i\eta}) \nonumber \\
    & + \frac{1}{2} \lambda^2 A^2 (e^{i\xi} - 2 + e^{-i\xi}) \nonumber \\
    & + \frac{1}{8} \lambda \mu  (AB+BA) (e^{i(\xi+\eta)} - e^{i(\xi-\eta)} - e^{-i(\xi-\eta)} + e^{-i(\xi+\eta)}) \nonumber \\
    & + \frac{1}{2} \mu^2 B^2 (e^{i\eta} - 2 + e^{-i\eta})
\end{align}
Formal wurde $[v^n_{j+p,k+q}]$ durch $e^{i(\xi p + \eta q)}$ ersetzt, wobei $-\pi \le \xi \le \pi$ und $-\pi \le \theta  \le \pi$. \newline

Unter Ausnutzung der Euler'schen Formel der Additionstheoreme für $\cos$ können wir $G$ umschreiben:
\begin{align}
    G = I 
    & + \frac{1}{2} \lambda A (2i \sin(\xi))+ \frac{1}{2} \mu B (2i \sin(\eta)) \nonumber \\
    & + \frac{1}{2} \lambda^2 A^2 (2 \cos(\xi) - 2) \nonumber \\
    & + \frac{1}{8} \lambda \mu  (AB+BA) (-4 \sin(\xi)\sin(\eta)) \nonumber \\
    & + \frac{1}{2} \mu^2 B^2 (2 \cos(\eta) - 2 ) \\
    \Leftrightarrow G =: & \phantom{ I + } (I-C) +iJ \nonumber
\end{align}
wobei
\begin{align*}
    C := \frac{1}{2} \lambda^2 A^2 (2 - 2 \cos(\xi))  + \frac{1}{8} \lambda \mu  (AB+BA) (4 \sin(\xi)\sin(\eta)) + \frac{1}{2} \mu^2 B^2 (2 - 2 \cos(\eta) )
\end{align*}
\begin{align*}
    J := \frac{1}{2} \lambda A (2 \sin(\xi))+ \frac{1}{2} \mu B (2 \sin(\eta))
\end{align*}
    

Es reicht nun aus, Stabilität für diesen Operator zu zeigen. Dabei ist $G \in Mat_{n,n}(\mathbb{C})$. Da alle Normen auf $Mat_{n,n}(\mathbb{C})$ äquivalent sind, können wir insbesondere den numerischen Radius als Norm verwenden. Wir wollen also zeigen:
\begin{align}
    w(G^n) \le 1
\end{align}
Dafür nutzen wir die "Power Inequality" (Theorem \ref{thm:power_ineq}). Dann ist es ausreichend zu zeigen, dass:
\begin{align}
    w(G)  \le 1
\end{align}
Wir folgen dem Beweis aus \parencite[][Theorem 4.5-1]{goldberg1982numerical}:

\begin{thm}(Stabilität)
    Das Lax-Wendroff-Verfahren (LW) ist stabil, falls für die Schrittweite $h_t$ gilt:
    \begin{align}
        & h_t  \le \frac{1}{2 \sqrt{ (\frac{w(A)}{h_x})^2 + (\frac{w(B)}{h_y})^2 } } \\
        \text{bzw.} \qquad & \lambda^2 w^2(A) + \mu^2 w^2(B) \le \frac{1}{4} \label{eq:ass_stab}
    \end{align}
\end{thm}
\begin{proof}
    Da $A$ und $B$ hermitesch sind, sind $I-C$ und $J$ auch hermitesch. Nach Lemma \ref{lem:nr_herm} gilt also $W(I-C), W(J) \in \mathbb{R}$ und somit für beliebige $x \in \mathbb{C}^n$ mit $\left\| x \right\| = 1$:
    \begin{align}
        |\left< Gx,x \right>|^2 & = |\left< (I-C)x,x \right>|^2+ |\left< Jx,x \right>|^2 \nonumber \\
        & = \left< (I-C)x,x \right>^2+ \left< Jx,x \right>^2 \nonumber \\
        & = (\left< Ix,x \right> - \left< Cx,x \right>)^2+ \left< Jx,x \right>^2 \nonumber \\
        & = 1 - 2 \left<Cx,x \right> + \left<Cx,x \right>^2 + \left< Jx,x \right>^2 \nonumber \\
        & = 1  + \underbrace{\left<Cx,x \right>^2 }_\text{$T_1$}  \underbrace{- 2 \left<Cx,x \right> + \left< Jx,x \right>^2}_\text{$T_2$} \label{eq:goal_stab}
    \end{align}
    Wenn wir zeigen können, dass für die Terme gilt $T_1+T_2 \le 0$, dann folgt 
    \begin{align}
        |\left< Gx,x \right>| \le 1 \Rightarrow w(G) \le 1
    \end{align}
    und mit der "Power Inequality" (Theorem \ref{thm:power_ineq}) die gewünschte Stabilität.
    \begin{enumerate}[label=\protect\circled{\arabic{*}}]
        \item Betrachten wir also zuerst den Term $T_2 = - 2 \left<Cx,x \right> + \left< Jx,x \right>^2$. Wir formen $C$ um zu:
        \begin{align}
            C &= \frac{1}{2} \lambda^2 A^2 (2 - 2 \cos(\xi)) + \frac{1}{8} \lambda \mu  (AB+BA) (4 \sin(\xi)\sin(\eta)) \nonumber \\
            & \phantom{mmm} + \frac{1}{2} \mu^2 B^2 (2 - 2 \cos(\eta) ) \nonumber \\
            \text{\tiny(bin. Formel)} &= \frac{1}{2} \biggl( \lambda^2 A^2 (2 - 2\cos(\xi)) + \mu^2 B^2 (2 - 2\cos(\eta) ) \nonumber \\
            & \phantom{mmm} + J^2 - \lambda^2  A^2 \sin^2(\xi) - \mu^2  B^2 \sin^2(\eta) \biggr) \nonumber \\
            &= \frac{1}{2} \biggl( \lambda^2 A^2 (2 - 2\cos(\xi) - \sin^2(\xi)) + \mu^2 B^2 (2 - 2\cos(\eta) - \sin^2(\eta)) \nonumber + J^2 \biggr) \nonumber \\
            \text{\tiny(Pythagoras)} &= \frac{1}{2} \biggl( \lambda^2 A^2 (1 - 2\cos(\xi) + \cos^2(\xi)) + \mu^2 B^2 (1 - 2\cos(\eta) + \cos^2(\eta)) + J^2 \biggr) \nonumber \\
            &= \frac{1}{2} \biggl( \lambda^2 A^2 (1 - \cos(\xi))^2+ \mu^2 B^2 (1 - \cos(\eta))^2 + J^2 \biggr) \nonumber \\
        \end{align}
        Dann folgt mit der Cauchy-Schwarz'schen Ungleichung:
        \begin{align}
            T_2 &= - 2 \left<Cx,x \right> + \left< Jx,x \right>^2 \nonumber \\
            &\le - 2 \left<Cx,x \right> + \left\| Jx \right\|^2 \left\| x \right\|^2 \nonumber \\
            &= - \left<\biggl( \lambda^2 A^2 (1 - \cos(\xi))^2 
            + \mu^2 B^2 (1 - \cos(\eta))^2 
            + J^2 \biggr)x,x \right> + \left\| Jx \right\|^2 \nonumber \\
            &= - \left< \lambda^2 A^2 (1 - \cos(\xi))^2 x,x \right> - \left< \mu^2 B^2 (1 - \cos(\eta))^2 x,x \right> - \left< J^2 x,x \right> + \left\| Jx \right\|^2 \nonumber \\
            &= -  \lambda^2 (1 - \cos(\xi))^2 \left< A^2  x,x \right> -  \mu^2 (1 - \cos(\eta))^2 \left< B^2  x,x \right> - \left< J^2 x,x \right> + \left\| Jx \right\|^2 \nonumber \\
            &= -  \lambda^2 (1 - \cos(\xi))^2 \left\| Ax \right\|^2 -  \mu^2 (1 - \cos(\eta))^2 \left\| Bx \right\|^2 - \left\| Jx \right\|^2 + \left\| Jx \right\|^2 \nonumber \\
            &= -  \lambda^2 (1 - \cos(\xi))^2 \left\| Ax \right\|^2 -  \mu^2 (1 - \cos(\eta))^2 \left\| Bx \right\|^2 \label{eq:final_eq2}
        \end{align}
        Dabei folgt die vorletzte Gleichheit, da $A,B,J$ hermitsch sind.
        \item Nun betrachten wir den Term $T_1=\left<Cx,x \right>^2$. Wir formen zunächst wieder um:
        \begin{align}
            \left<Cx,x \right> = \lambda^2 (1 - \cos(\xi)) \left< A^2x,x \right> & + \frac{1}{2} \lambda \mu (\sin(\xi)\sin(\eta)) \left< (AB+BA)x,x \right> \nonumber \\
            & + \mu^2 (1 - \cos(\eta) ) \left<B^2x,x \right> \nonumber \\
            = \lambda^2 (1 - \cos(\xi)) \left\| Ax\right\|^2 & + \mu^2 (1 - \cos(\eta) ) \left\| Bx \right\|^2 \nonumber \\
            & + \frac{1}{2} \lambda \mu (\sin(\xi)\sin(\eta)) \left< (AB+BA)x,x \right> \nonumber \\
        \end{align}
        Den letzen Term schätzen wir wieder unter Ausnutzung, dass $A,B$ hermitesch sind, nach oben ab durch:
        \begin{align}
            &\phantom{....} \left| \frac{1}{2} \lambda \mu (\sin(\xi)\sin(\eta)) \left< (AB+BA)x,x \right> \right|\nonumber \\
            &=  \left|\frac{1}{2} \lambda \mu (\sin(\xi)\sin(\eta)) \left[ \left< ABx,x \right> + \left< BAx,x \right>\right] \right| \nonumber \\
            &= \left| \frac{1}{2} \lambda \mu (\sin(\xi)\sin(\eta)) \left[ \left< Bx,Ax \right> + \left< Ax,Bx \right>\right] \right| \nonumber \\
            &\le \left| \lambda \mu (\sin(\xi)\sin(\eta)) \left< Bx,Ax \right> \right| \nonumber \\
            & \overset{CS}{\le} \left| \lambda \mu (\sin(\xi)\sin(\eta)) \right|  \left\| Bx \right\| \left\| Ax \right\|\nonumber \\
            \text{(\tiny{bin. Formel})}&\le \frac{1}{2} \left( \lambda^2 \sin^2\xi \left\| Ax \right\|^2 + \mu^2 \sin^2\eta \left\| Bx \right\|^2 \right) \nonumber \\
            &= \frac{1}{2} \left( \lambda^2 (1-\cos^2\xi) \left\| Ax \right\|^2 + \mu^2 (1-\cos^2\eta) \left\| Bx \right\|^2 \right) \nonumber \\
            & \overset{*}{\le}  \lambda^2 (1-\cos\xi) \left\| Ax \right\|^2 + \mu^2 (1-\cos\eta) \left\| Bx \right\|^2 \nonumber \\
        \end{align}
        Wobei $*$ aus der Gleichung
        \begin{align*}
            \frac{1}{2} (1 -\cos^2 \phi) = \underbrace{\frac{1}{2} (1 + \cos \phi)}_{\le 1} (1- \cos \phi) \le 1 - \cos \phi
        \end{align*}
        folgt. Also gilt 
        \begin{align}
            \left| \left<Cx,x \right> \right| \le 2\lambda^2 (1-\cos\xi) \left\| Ax \right\|^2 + 2\mu^2 (1-\cos\eta) \left\| Bx \right\|^2
        \end{align} 
        Schließlich können wir den Term $T_1$ nach oben abschätzen durch:
        \begin{align}
            T_1 &= \left<Cx,x \right>^2 \nonumber \\ 
            &\le 4 \left( \lambda^2 (1-\cos\xi) \left\| Ax \right\|^2 + \mu^2 (1-\cos\eta) \left\| Bx \right\|^2 \right)^2 \nonumber \\
            &= 4 \biggl( \lambda^4 (1-\cos\xi)^2 \left\| Ax \right\|^4 + \mu^4 (1-\cos\eta)^2 \left\| Bx \right\|^4 \nonumber \\[-10pt]
            &\phantom{===} + 2(1-\cos \xi)(1-\cos \eta) \lambda^2 \left\| Ax \right\|^2 \mu^2 \left\| Bx \right\|^2 \biggr) \nonumber \\
            &\overset{**}{\le} 4 \biggl( \lambda^4 (1-\cos\xi)^2 \left\| Ax \right\|^4 + \mu^4 (1-\cos\eta)^2 \left\| Bx \right\|^4 \nonumber \\[-10pt] 
            &\phantom{===} + 2\left[(1-\cos \xi)^2+(1-\cos \eta)^2\right] \lambda^2 \left\| Ax \right\|^2 \mu^2 \left\| Bx \right\|^2 \biggr) \nonumber \\
            &=4 \biggl( \lambda^2 \left\| Ax \right\|^2 \bigl[ (1-\cos \xi)^2\lambda^2 \left\| Ax \right\|^2 + (1-\cos \xi)^2 \mu^2 \left\| Bx \right\|^2 \bigr] \nonumber \\[-10pt]
            &\phantom{===} + \mu^2 \left\| Bx \right\|^2 \bigl[ (1-\cos \eta)^2\mu^2 \left\| Bx \right\|^2 + (1-\cos \xi)^2 \lambda^2 \left\| Ax \right\|^2 \bigr] \biggr) \nonumber \\
            &= 4 \bigl(\lambda^2 \left\| Ax \right\|^2 + \mu^2 \left\| Bx \right\|^2 \bigr) \bigl((1-\cos \xi)^2\lambda^2 \left\| Ax \right\|^2 + (1-\cos \xi)^2 \mu^2 \left\| Bx \right\|^2 \bigr)  \nonumber \\
            & \overset{***}{\le} \bigl((1-\cos \xi)^2\lambda^2 \left\| Ax \right\|^2 + (1-\cos \xi)^2 \mu^2 \left\| Bx \right\|^2 \bigr) \\ \label{eq:final_ineq_stab}
        \end{align}

        Wobei $**$ aus einer Abschätzung mithilfe der binomischen Formel folgt
        \begin{align}
            2ab \le 2ab + (a-b)^2 = a^2 + b^2 \le 2a^2 + 2 b^2            
        \end{align}
        
        
        und $***$ aus der Annahme aus Gleichung (\ref{eq:ass_stab}), sowie der Bemerkung \ref{rem:nr_herm} über die Gleichheit von Spektralradius, numerischen Radius und Operatornorm bei hermiteschen Matrizen:
        \begin{align}
            4 \bigl(\lambda^2 \left\| Ax \right\|^2 + \mu^2 \left\| Bx \right\|^2 \bigr) \overset{CS}&{\le} 4 \bigl(\lambda^2 \left\| A \right\|^2 + \mu^2 \left\| B \right\|^2 \bigr) \nonumber \\
            \text{\tiny (Bem. \ref{rem:nr_herm}) }&= 4 \bigl(\lambda^2 w^2(A) + \mu^2 w^2(B) \bigr) \nonumber \\
            \overset{(\ref{eq:ass_stab})}&{\le} 1 \qquad .
        \end{align}
    \end{enumerate}
    Schließlich erhalten wir insgesamt aus den Gleichungen (\ref{eq:final_eq2}) und (\ref{eq:goal_stab})
    \begin{align}
        |\left< Gx,x \right>|^2 &= 1  + T_1 + T_2 \overset{(\ref{eq:final_ineq_stab})}{\le}  1 \qquad .
    \end{align} 
    Also $w(G) \le 1$ und mit der Power-Inequality (Theorem \ref{thm:power_ineq}) folgt $w(G^n) \le 1$. Damit ist $G$ und somit das Lax-Wendroff-Schema stabil.
\end{proof}

Die Konvergenz des Lax-Wendroff-Verfahrens ergibt sich nun als Korollar mithilfe des Äquivalenzsatzes von Lax (Theorem \ref{thm:equival_lax}), da wir Stabilität und Konsistenz des Verfahrens gezeigt haben.

\begin{cor}(Konvergenz des Lax-Wendroff-Verfahrens)
    Das Lax-Wendroff-Verfahren (LW) ist konvergent, falls für die Schrittweite $h_t$ gilt:
    \begin{align}
        & h_t  \le \frac{1}{2 \sqrt{ (\frac{w(A)}{h_x})^2 + (\frac{w(B)}{h_y})^2 } } \\
        \text{bzw.} \qquad & \lambda^2 w^2(A) + \mu^2 w^2(B) \le \frac{1}{4}
    \end{align}
\end{cor}
